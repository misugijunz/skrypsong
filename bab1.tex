%-----------------------------------------------------------------------------%
\chapter{\babSatu}
%-----------------------------------------------------------------------------%
\todo{tambahkan kata-kata pengantar bab 1 disini}


%-----------------------------------------------------------------------------%
\section{Latar Belakang}
%-----------------------------------------------------------------------------%
\todo{tuliskan latar belakang penelitian disini}
Teknologi \f{internet} berkembang pesat dalam dekade terakhir. Perkembangan yang dapat dirasakan adalah memungkinka orang-orang dalam satu tim untuk bekerja dan berkolaborasi Tak dapat dipungkiri hal ini mendorong pesatnya proyek-proyek pengembangan aplikasi terutama yang berbasis \f{web}. 


%-----------------------------------------------------------------------------%
\section{Permasalahan}
%-----------------------------------------------------------------------------%
Pada bagian ini akan dijelaskan mengenai definisi permasalahan 
yang \saya~hadapi dan ingin diselesaikan serta asumsi dan batasan 
yang digunakan dalam menyelesaikannya.


%-----------------------------------------------------------------------------%
\subsection{Definisi Permasalahan}
%-----------------------------------------------------------------------------%
\todo{Tuliskan permasalahan yang ingin diselesaikan. Bisa juga
	berbentuk pertanyaan}


%-----------------------------------------------------------------------------%
\subsection{Batasan Permasalahan}
%-----------------------------------------------------------------------------%
\todo{Umumnya ada asumsi atau batasan yang digunakan untuk 
	menjawab pertanyaan-pertanyaan penelitian diatas.}


%-----------------------------------------------------------------------------%
\section{Tujuan}
%-----------------------------------------------------------------------------%
\todo{Tuliskan tujuan penelitian.}


%-----------------------------------------------------------------------------%
\section{Posisi Penelitian}
%-----------------------------------------------------------------------------%
\todo{Posisi penelitian Anda jika dilihat secara bersamaan dengan 
	peneliti-peneliti lainnya. Akan lebih baik lagi jika ikut menyertakan 
	diagram yang menjelaskan hubungan dan keterkaitan antar 
	penelitian-penelitian sebelumnya}


%-----------------------------------------------------------------------------%
\section{Metodologi Penelitian}
%-----------------------------------------------------------------------------%
\todo{Tuliskan metodologi penelitian yang digunakan.}
Penelitian tugas akhir ini meliputi pelaksanaan tahapan-tahapan sebagai berikut: 
\begin{enumerate}
\item Studi Literatur

\item Pendefinisian Metodologi, Metode, Alat Bantu, dan Proses

\item Pengumpulan dan Analisis Kebutuhan Perangkat Lunak Bantu \f{\Reqm}

\item Perancangan Prototipe (Implementasi) Perangkat Lunak Bantu \f{\Reqm} 

\item Pengembangan Prototipe (Implementasi) Perangkat Lunak Bantu \f{\Reqm}

\item Uji Coba dan Analisis Prototipe
 
\end{enumerate}

%-----------------------------------------------------------------------------%
\section{Sistematika Penulisan}
%-----------------------------------------------------------------------------%
Sistematika penulisan laporan adalah sebagai berikut:
\begin{itemize}
	\item Bab 1 \babSatu \\
	\item Bab 2 \babDua \\
	\item Bab 3 \babTiga \\
	\item Bab 4 \babEmpat \\
	\item Bab 5 \babLima \\
	\item Bab 6 \kesimpulan \\
\end{itemize}

\todo{Tambahkan penjelasan singkat mengenai isi masing-masing bab.}
Bab 1

Bab 2

Bab 3

Bab 4

Bab 5

Bab 6

